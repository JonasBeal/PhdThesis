%%%%%%%%%%%%%%%%%%%%%%%%%%%%%%%%%%%%%%%%%%%%%%%%%%%%%%%%%%%%%
%Start preamble
%%%%%%%%%%%%%%%%%%%%%%%%%%%%%%%%%%%%%%%%%%%%%%%%%%%%%%%%%%%%%

\usepackage{booktabs}
\usepackage{amsthm}
\usepackage{./cover/psl-cover} % specifies the path to the cover page template
%\usepackage{psl-cover} % specifies the path to the cover page template

%\makeatletter
%\def\thm@space@setup{%
%  \thm@preskip=8pt plus 2pt minus 4pt
%  \thm@postskip=\thm@preskip
%}
%\makeatother


%%%%%%%%%%%%%%%%%%%%%%%%%%%%%%%%%%%%%%%%%%%%%%%%%%%%%%%%%%%%%
% Add a lettrine to the very first character of the content %
%%%%%%%%%%%%%%%%%%%%%%%%%%%%%%%%%%%%%%%%%%%%%%%%%%%%%%%%%%%%%

\usepackage{lettrine} % supports various dropped capitals styles

\newcommand{\initial}[1]{
	\lettrine[lines=3,lhang=0.33,nindent=0em]{
		\color{gray}
     		{\textsc{#1}}}{}}
     		

%%%%%%%%%%%%%%%%%%%%%%%%%%%%%%%%%%%%%%%%%%%%%%%%%%%%%%%%%%%%%
% Links box
%%%%%%%%%%%%%%%%%%%%%%%%%%%%%%%%%%%%%%%%%%%%%%%%%%%%%%%%%%%%%
\usepackage{hyperref}
\hypersetup{
    colorlinks=true,
    linkcolor=black,
    filecolor=blue,      
    urlcolor=blue,
    citecolor=black
}

%%%%%%%%%%%%%%%%%%%%%%%%%%%%%%%%%%%%%%%%%%%%%%%%%%%%%%%%%%%%%
% Text box
%%%%%%%%%%%%%%%%%%%%%%%%%%%%%%%%%%%%%%%%%%%%%%%%%%%%%%%%%%%%%
%\usepackage{xcolor}
%\usepackage[dvipsnames]{xcolor}
%\PassOptionsToPackage{usenames,dvipsnames}{xcolor}
\usepackage{xcolor}
\definecolor{steelblue}{rgb}{0.27, 0.51, 0.71}
\definecolor{maroon}{rgb}{0.5, 0.0, 0.0}
\definecolor{indianred}{rgb}{0.8, 0.36, 0.36}

\usepackage{tcolorbox}
\newtcolorbox{summarybox}{
  colback= black!10!white,
  colframe=steelblue,
  coltext=black,
  boxsep=5pt,
  arc=4pt}
  
\newtcolorbox{conclubox}{
  colback= black!10!white,
  colframe=indianred,
  coltext=black,
  boxsep=5pt,
  arc=4pt}
     		
%%%%%%%%%%%%%%%%%%%%%%%%
% Insert an empty page %
%%%%%%%%%%%%%%%%%%%%%%%%

\usepackage{afterpage} % executes command after the next page break

\newcommand\blankpage{%
    \null
    \thispagestyle{empty}%
    % \addtocounter{page}{-1}% % uncomment to increase page counter
    \newpage
    }

\newcommand{\clearemptydoublepage}{\newpage{\thispagestyle{empty}\cleardoublepage}}

%%%%%%%%%%%%%%%%%%
% Epigraph style %
%%%%%%%%%%%%%%%%%%

\usepackage{epigraph} % provides commands to assist in the typesetting of a single epigraph

\setlength\epigraphwidth{1\textwidth}
\setlength\epigraphrule{0pt} % no line between
\setlength\beforeepigraphskip{1\baselineskip} % space before and after epigraph
\setlength\afterepigraphskip{2\baselineskip}
\renewcommand*{\textflush}{flushright}
\renewcommand*{\epigraphsize}{\normalsize\itshape}

%%%%%%%%%%%%%%%%%%%%%%%%
% Formatting
%%%%%%%%%%%%%%%%%%%%

\usepackage{calc} % simple arithmetics in latex commands
\usepackage{soul} % hyphenation for letterspacing, underlining, etc.
\renewcommand{\topfraction}{.85}
\setlength{\parskip}{1em}

\makeatletter
\newlength\dlf@normtxtw
\setlength\dlf@normtxtw{\textwidth}
\newsavebox{\feline@chapter}
\newcommand\feline@chapter@marker[1][4cm]{%
	\sbox\feline@chapter{%
		\resizebox{!}{#1}{\fboxsep=1pt%
			\colorbox{gray}{\color{white}\thechapter}%
		}}%
		\rotatebox{90}{%
			\resizebox{%
				\heightof{\usebox{\feline@chapter}}+\depthof{\usebox{\feline@chapter}}}%
			{!}{\scshape\so\@chapapp}}\quad%
		\raisebox{\depthof{\usebox{\feline@chapter}}}{\usebox{\feline@chapter}}%
}

\newcommand\feline@chm[1][4cm]{%
	\sbox\feline@chapter{\feline@chapter@marker[#1]}%
	\makebox[0pt][c]{% aka \rlap
		\makebox[1cm][r]{\usebox\feline@chapter}%
	}}

\makechapterstyle{daleifmodif}{
\renewcommand\chapnamefont{\normalfont\Large\scshape\raggedleft\so}
\renewcommand\chaptitlefont{\normalfont\Large\bfseries\scshape}
\renewcommand\chapternamenum{} \renewcommand\printchaptername{}
\renewcommand\printchapternum{\null\hfill\feline@chm[2.5cm]\par}
\renewcommand\afterchapternum{\par\vskip\midchapskip}
\renewcommand\printchaptertitle[1]{\color{gray}\chaptitlefont\raggedleft
  \renewcommand\chaptername{Chapter}
  ##1\par}
}

\makeatother
\chapterstyle{daleifmodif}

% The pages should be numbered consecutively at the bottom centre of the page
\makepagestyle{myvf}
\makeoddfoot{myvf}{}{\thepage}{}
\makeevenfoot{myvf}{}{\thepage}{}
\makeheadrule{myvf}{\textwidth}{\normalrulethickness}
\makeevenhead{myvf}{\small\textsc{\leftmark}}{}{}
\makeoddhead{myvf}{}{}{\small\textsc{\rightmark}}
\pagestyle{myvf}


%%%%%%%%%%%%%%%%%%%%%%%%%%%%%%%%%%%%%%%%%%%%%%%%%%%%%%%%%%%%%
% Define cover page settings
%%%%%%%%%%%%%%%%%%%%%%%%%%%%%%%%%%%%%%%%%%%%%%%%%%%%%%%%%%%%%

\title{My title}

\author{Jonas BEAL}

\institute{l'Institut Curie}
\doctoralschool{Complexite du Vivant}{515}

\institute{l'Institut Curie}
\doctoralschool{Complexité du Vivant}{515}
\specialty{Génomique}
\date{23 septembre 2020}

%% cotutelle
% \entitle{Thesis Subject in English}
% \otherinstitute{CEA Saclay}
% \logootherinstitute{logo-institute}

\jurymember{1}{Adeline LECLERQ-SAMSON}{Professeur, Université Grenoble Alpes}{Rapporteur}
\jurymember{2}{Lodewyk WESSELS}{Professeur, Netherlands Cancer Institute}{Rapporteur}
\jurymember{3}{Émilie LANOY}{Ingénieure de recherche, Gustave Roussy}{Examinateur}
\jurymember{4}{Denis THIEFFRY}{Professeur, ENS, PSL}{Examinateur}
\jurymember{5}{Emmanuel BARILLOT}{Directeur de Recherche, Institut Curie, PSL}{Directeur de thèse}
\jurymember{6}{Aurélien LATOUCHE}{Professeur, Institut Curie, Cnam}{Directeur de thèse}
\jurymember{7}{Laurence CALZONE}{Ingénieure de recherche, Institut Curie, PSL}{Membre invitée, co-encadrante de thèse}


\frabstract{
  Au delà de ses mécanismes génétiques, le cancer peut être compris comme une maladie de réseaux qui résulte souvent de l’interaction entre différentes perturbations dans un réseau de régulation cellulaire.  La dynamique de ces réseaux et des voies de signalisation associées est complexe et requiert des approches intégrées. Une d’entre elles est la conception de modèles dits mécanistiques qui traduisent mathématiquement la connaissance biologique des réseaux afin de pouvoir simuler le fonctionnement moléculaire des cancers informatiquement. Ces modèles ne traduisent cependant que les mécanismes généraux à l’oeuvre dans certains cancers en particulier.  
  

Cette thèse propose en premier lieu de définir des modèles mécanistiques personnalisés de cancer. Un modèle générique est  d’abord défini dans un formalisme logique (ou Booléen), avant d’utiliser les données omiques (mutations, ARN, protéines) de patients ou de lignées cellulaires afin de rendre le modèle spécifique à chacun. Ces modèles personnalisés peuvent ensuite être confrontés aux données cliniques de patients pour vérifier leur validité. Le cas de la réponse clinique aux traitements est exploré en particulier dans ce travail. La représentation explicite des mécanismes moléculaires par ces modèles permet en effet de simuler l’effet de différents traitements suivant leur mode d’action et de vérifier si la sensibilité d’un patient à un traitement est bien prédite par le modèle personnalisé correspondant. Un exemple concernant la réponse aux inhibiteurs de BRAF dans les mélanomes et cancers colorectaux est ainsi proposé.  
  

La confrontation des modèles mécanistiques de cancer, ceux présentés dans cette thèse et d’autres, aux données cliniques incite par ailleurs à évaluer rigoureusement leurs éventuels bénéfices dans la cadre d’une utilisation médicale. La quantification et l’interprétation de la valeur pronostique des biomarqueurs issus de certains modèles méchanistiques est brièvement présentée avant de se focaliser sur le cas particulier des modèles capables de sélectionner le meilleur traitement pour chaque patient en fonction des ses caractéristiques moléculaires. Un cadre théorique est proposé pour étendre les méthodes d’inférence causale à l’évaluation de tels algorithmes de médecine de précision. Une illustration est fournie à l’aide de données simulées et de xénogreffes dérivées de  patients.  
  

L’ensemble des méthodes et applications décrites tracent donc un chemin, de la conception de modèles mécanistiques de cancer à leur évaluation grâce à des modèles statistiques émulant des essais cliniques, proposant ainsi un cadre pour la mise en oeuvre de la médecine de précision en oncologie.

}

\enabstract{
  Beyond its genetic mechanisms, cancer can be understood as a network disease that often results from the interactions between different perturbations in a cellular regulatory network.  The dynamics of these networks and associated signaling pathways are complex and require integrated approaches. One approach is to design mechanistic models that translate the biological knowledge of networks in mathematical terms to simulate computationally the molecular features of cancers. However, these models only reflect the general mechanisms at work in cancers.  
  

This thesis proposes to define personalized mechanistic models of cancer. A generic model is first defined in a logical (or Boolean) formalism, before using omics data (mutations, RNA, proteins) from patients or cell lines in order to make the model specific to each one profile. These personalized models can then be compared with the clinical data of patients in order to validate them. The response to treatment is investigated in particular in this thesis. The explicit representation of the molecular mechanisms by these models allows to simulate the effect of different treatments according to their targets and to verify if the sensitivity of a patient to a drug is well predicted by the corresponding personalized model. An example concerning the response to BRAF inhibitors in melanomas and colorectal cancers is thus presented.  
  

The comparison of mechanistic models of cancer, those presented in this thesis and others, with clinical data also encourages a rigorous evaluation of their possible benefits in the context of medical use. The quantification and interpretation of the prognostic value of outputs of some mechanistic models is briefly presented before focusing on the particular case of models able to recommend the best treatment for each patient according to his molecular profile. A theoretical framework is defined to extend causal inference methods to the evaluation of such precision medicine algorithms. An illustration is provided using simulated data and patient derived xenografts.  
  

All the methods and applications put forward a possible path from the design of mechanistic models of cancer to their evaluation using statistical models emulating clinical trials. As such, this thesis provides one framework for the implementation of precision medicine in oncology.

}

\frkeywords{ Modélisation, Cancer, Modèle mécanistique, Biostatistiques, Inférence causale, Médecine de précision.}
\enkeywords{ Modeling, Cancer, Mechanistic model, Biostatistics, Causal inference, Precision medicine.}
  
\pagenumbering{roman}